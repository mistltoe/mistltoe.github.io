\subsection{Numerical Array}
\label{subsec:numericalarray}
Array could also contain numerical values, and this way of usage is more common when you do image analysis. We examine a simple example of using numerical array that prints out intensity profile along selected line ROI. We use\ilcom{getProfile} function to access intensity profile. 

\begin{shaded}\begin{indentCom}
\textbf{getProfile()}\\
Runs Analyze>Plot Profile (without displaying the plot) and returns the intensity values as an array. For an example, see the GetProfileExample macro\footnote{\url{http://rsb.info.nih.gov/ij/macros/GetProfileExample.txt}}. See also: Plot.getValues().
\end{indentCom}\end{shaded}

Before running code20\_5.ijm, an image (could be anything) a line ROI should be present in the active image.

\lstinputlisting[morekeywords={*, newArray, selectionType, getProfile, setResult, updateResults}]{code/code20_5.ijm}

\begin{itemize}
\item Line 3: Check if the selection type is a straight line ROI. If not, macro terminates leaving a message. 

\begin{indentCom}
\textbf{selectionType}()\\ 
Returns the selection type, where 0=rectangle, 1=oval, 2=polygon, 3=freehand, 4=traced, 5=straight line, 6=segmented line, 7=freehand line, 8=angle, 9=composite and 10=point. Returns -1 if there is no selection.
\end{indentCom}

\item Line 4: Empty array \ilcom{tempProfile} is loaded with the intensity profile along the line ROI by \ilcom{getProfile}().
\item
\begin{indentCom}
\textbf{getProfile}()\\
Runs \ijmenu{[Analyze > Plot Profile]} (without displaying the plot) and returns the intensity values as an array.
\end{indentCom}

\item Line 5: Passing the array \ilcom{tempProfile} to function "output\_results", which prints the content of array in the table shown in the ``Results'' window. 

\item Line 7 to 14: A function for outputting the profile array in the table shown in the ``Results'' window. It takes an argument \ilcom{rA}, which is supposed to be an array. 
\item Line 8: Clears the results table. 
\item Line 9 to 12: for-loop to go through the array and to print out each element. 
\item Line 10: Sets the pixel position along the segment in the column labeled "n". 
\item Line 11: Sets the content of the array (pixel intensity) in the column labeled "intensity".

\begin{indentCom}
\textbf{setResult}("Column", row, value)
Adds an entry to the ImageJ results table or modifies an existing entry. The first argument specifies a column in the table. If the specified column does not exist, it is added. The second argument specifies the row, where 0<=row<=nResults. (nResults is a predefined variable.) A row is added to the table if row=nResults. The third argument is the value to be added or modified. 
\end{indentCom}
\item Line 13: Updates the table shown in the ``Results'' window. 

\begin{indentCom}
\textbf{updateResults}()
Call this function to update the "Results" window after the results table has been modified by calls to the setResult() function. 
\end{indentCom}
\end{itemize}

\begin{indentexercise}{1}
Modify code 20.5 that the macro calculates the sum of all intensities.\\

Hint:

\begin{itemize}
\item
\item You do not need the function anymore. 
\item \ilcom{for}-loop should be used.
\item Use \ilcom{tempProfile.length}
\end{itemize}

\item \textbf{Answer}: The key for getting the sum of values in an array is for-loop to go through all elements of the array. The total sum of array values is calculated by adding up values during this for-loop.   
\begin{lstlisting}[numbers=none]
macro "get profile and printout" {
	if (selectionType() !=5) exit("selection type must be a straight line ROI");
	tempProfile=getProfile();
	sum = 0;
	for (i = 0; i < tempProfile.length; i++){
		sum += tempProfile[i];
	}
	print("sum of values:", sum);
}  
\end{lstlisting}

Another way of achieving the similar task is by using Array related function. We will see this later. 
\end{indentexercise}
